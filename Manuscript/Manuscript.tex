%% Copernicus Publications Manuscript Preparation Template for LaTeX Submissions
%% ---------------------------------
%% This template should be used for copernicus.cls
%% The class file and some style files are bundled in the Copernicus Latex Package, which can be downloaded from the different journal webpages.
%% For further assistance please contact Copernicus Publications at: production@copernicus.org
%% https://publications.copernicus.org/for_authors/manuscript_preparation.html

%% copernicus_rticles_template (flag for rticles template detection - do not remove!)

%% Please use the following documentclass and journal abbreviations for discussion papers and final revised papers.

%% 2-column papers and discussion papers
\documentclass[gc, manuscript]{copernicus}



%% Journal abbreviations (please use the same for preprints and final revised papers)

% Advances in Geosciences (adgeo)
% Advances in Radio Science (ars)
% Advances in Science and Research (asr)
% Advances in Statistical Climatology, Meteorology and Oceanography (ascmo)
% Annales Geophysicae (angeo)
% Archives Animal Breeding (aab)
% Atmospheric Chemistry and Physics (acp)
% Atmospheric Measurement Techniques (amt)
% Biogeosciences (bg)
% Climate of the Past (cp)
% DEUQUA Special Publications (deuquasp)
% Drinking Water Engineering and Science (dwes)
% Earth Surface Dynamics (esurf)
% Earth System Dynamics (esd)
% Earth System Science Data (essd)
% E&G Quaternary Science Journal (egqsj)
% EGUsphere (egusphere) | This is only for EGUsphere preprints submitted without relation to an EGU journal.
% European Journal of Mineralogy (ejm)
% Fossil Record (fr)
% Geochronology (gchron)
% Geographica Helvetica (gh)
% Geoscience Communication (gc)
% Geoscientific Instrumentation, Methods and Data Systems (gi)
% Geoscientific Model Development (gmd)
% History of Geo- and Space Sciences (hgss)
% Hydrology and Earth System Sciences (hess)
% Journal of Bone and Joint Infection (jbji)
% Journal of Micropalaeontology (jm)
% Journal of Sensors and Sensor Systems (jsss)
% Magnetic Resonance (mr)
% Mechanical Sciences (ms)
% Natural Hazards and Earth System Sciences (nhess)
% Nonlinear Processes in Geophysics (npg)
% Ocean Science (os)
% Polarforschung - Journal of the German Society for Polar Research (polf)
% Primate Biology (pb)
% Proceedings of the International Association of Hydrological Sciences (piahs)
% Safety of Nuclear Waste Disposal (sand)
% Scientific Drilling (sd)
% SOIL (soil)
% Solid Earth (se)
% The Cryosphere (tc)
% Weather and Climate Dynamics (wcd)
% Web Ecology (we)
% Wind Energy Science (wes)

% Pandoc citation processing

% The "Technical instructions for LaTex" by Copernicus require _not_ to insert any additional packages.
% 
% tightlist command for lists without linebreak
\providecommand{\tightlist}{%
  \setlength{\itemsep}{0pt}\setlength{\parskip}{0pt}}


%
\begin{document}


\title{New insights into the Weddell Sea ecosystem applying a
quantitative network approach}


\Author[1, *]{Tomás I.}{Marina}
\Author[1, *]{Leonardo A.}{Saravia}
\Author[2]{Susanne}{Kortsch}


\affil[1]{Centro Austral de Investigaciones Científicas (CADIC-CONICET),
Ushuaia, Argentina}
\affil[2]{Spatial Foodweb Ecology Group, Department of Agricultural
Sciences, University of Helsinki, Finland}
\affil[*]{These authors contributed equally to this work.}

\runningtitle{New insights into the Weddell Sea food web}

\runningauthor{Marina et al.}

\correspondence{Tomás I. Marina (tomasimarina@gmail.com) and Leonardo A.
Saravia (arysar@gmail.com)}


\received{}
\pubdiscuss{} %% only important for two-stage journals
\revised{}
\accepted{}
\published{}

%% These dates will be inserted by Copernicus Publications during the typesetting process.


\firstpage{1}

\maketitle


\begin{abstract}
Network approaches can shed light on the structure and stability of
complex marine communities. In recent years, such approaches have been
successfully applied to study polar ecosystems, improving our knowledge
on how they might respond to ongoing environmental changes. The Weddell
Sea is one of the most studied marine ecosystems outside the Antarctic
Peninsula in the Southern Ocean. Yet, few studies consider the known
complexity of the Weddell Sea food web, which in its current form
comprises 490 species and 16041 predator-prey interactions. Here we
analysed the Weddell Sea food web, focusing on the species and trophic
interactions that underpin ecosystem structure and stability. We
estimated the strength for each interaction in the food web,
characterised species position in the food web using unweighted and
weighted food web properties, and analysed species' roles with respect
to the stability of the food web. We found that the distribution of the
interaction strength at the food web level is asymmetric, with many weak
interactions and few strong ones. We detected a positive relationship
between species mean interaction strength and two unweighted properties
(i.e., trophic level and the total number of interactions). We also
found that only a few species possess key positions in terms of food web
stability. These species are characterised by high mean interaction
strength, mid to high trophic level, relatively high number of
interactions, and mid to low trophic similarity. In this study, we
integrated unweighted and weighted food web information, enabling a more
complete assessment of the ecosystem structure and function of the
Weddell Sea food web. Our results provide new insights, which are
important for the development of effective policies and management
strategies, particularly given the ongoing initiative to implement a
Marine Protected Area (MPA) in the Weddell Sea.
\end{abstract}




\section{Introduction}

Food web analysis constitutes an important framework for understanding
ecological community structure and for conserving biodiversity through
ecosystem management \citep{Thompson2012}. Although topological food web
analysis, which considers only the presence and absence of predator-prey
interactions, provides important insights into the structure and
functioning of ecological communities
\citep[e.g.][]{Pascual2006, Kortsch2015, Marina2018, Cordone2020, Rodriguez2022},
more information on the nature of the trophic interactions is needed to
effectively charaterise ecosystem dynamics and stability
\citep{Kortsch2021}. This is a fundamental step for providing
assemessments on ecosystem vulnerability to environmental pressures and
for prioritising management actions. In this regard, quantifying the
strength of trohic interactions and species' roles within the network
are of paramount importance
\citep{Carrara2015, Allesina2015, Nilsson2016, Cirtwill2018a}.

Estimating interaction strength in food webs allows differentiating the
importance of species interactions. On the contrary, unweighetd food web
representations give equal importantance to all interactions
well-knowing that some species interactions are stronger than others and
hence play a diffrent role for ecosystem functioning and stability. Both
empirical and theoretical studies show that interactions strength
distributions in food webs are assymetric
\citep{Paine1992, McCann1998, Emmerson2004, Wootton2005, Kortsch2021},
containing a few strong and many weak interactions. This asymmetric
patterning of weak and strong links is crucial to food web stability
\citep{Paine1992, McCann1998, Neutel2002}. In a recent paper on an
aquatic food web it was further highlighted that temporal changes in
ecosytem functioning could only be predicted using weighted food web
structure \citep{Kortsch2021}. Hence, in order to assess the stability
and functioning of a food web, it is important to first determine the
quantitative structure of the trophic network.

Several methodologies have been applied to estimate interaction strength
in food webs, where the quantity and quality of the data mostly
determines which approach is the most convenient \citep{Berlow2004}.
Approaches include experimental methods combined with dynamic modelling
\citep{Emmerson2004, Carrara2015}, measurements of species abundances
through time \citep{Fahimipour2014, Chang2021}, and estimation of
metabolic rates and biomass of all species in the community
\citep{Neutel2014}. However, these types of methods require large
experimental set-ups and parameterisations restricting the analyses to
smaller networks (e.g., approximatley 10 species or less). Other methods
based on allometric scaling relationships and biomass information
\citep{Kortsch2021, Gauzens2019} can be applied to larger networks with
less data requirements, but this comes at the expense of precision in
the predictions. For even larger food webs composed of nearly 1000
species and more than 10000 interactions, only methods with even less
data requirements are feasible. One of these methods, proposed by
\citet{Pawar2012}, combines data on consumer and resource body masses
and consumer search space (interaction dimensionality) to obtain
interaction strength estimates for each pairwise predator-prey
interaction. An advantage of this method is that it can be applied
without information on species biomass.

Using a a network approach, different types (e.g., terrestial, lake,
marine) of food webs from various geographic locations are studied
worldwide, including marine polar food webs
\citep{Carscallen2012, Santana2013, Kortsch2019, Pecuchet2022}. Some of
the studies from the Arctic show how food web properties (e.g.,
connectance) are constrained by environmental factors such as sea ice
cover and seawater temperature \citep{Kortsch2019, Pecuchet2022}. In the
Southern Ocean, important insights have been gained into mechanisms of
energy flow, the relative importance of individual species and their
traits, and the influence of environmental variables (e.g., sea-ice) on
the structure of local food webs \citep{Cordone2020, Rossi2019}. For
instance, in Potter Cove (West Antarctic Peninsula) the substratum type
(i.e.~hard/soft or rocks/sediments) plays a significant role in the
structure and stability of the food web. In Terra Nova Bay (Ross Sea),
the architecture of biodiversity was reshaped by the pulsed input of
sympagic food sources following sea-ice break up, with food web
simplification, decreased intraguild predation, potential disturbance
propagation and increased vulnerability to biodiversity loss
\citep{Rossi2019}.

The Weddell Sea is expected to be one of the last regions of the
Southern Ocean to experience the consequences of climate change due to
its extensive ice cover and ocean currents \citep{Teschke2021} resulting
in less sea surface warming compared to other areas of the Southern
Ocean. This Sea plays an important role in driving global thermohaline
circulation and ventilating the global abyssal ocean because it
generates a considerable part of the Antarctic Bottom Water
\citep{Fahrbach2009}. Because of these environmental characteristics,
the Weddell Sea may serve as a refuge for Antarctic species which depend
on sea ice (e.g.~krill, emperor penguin, Weddell seal) or have low heat
tolerance (e.g., most notothenioid fishes) due to their adaptations to
freezing tempeartures \citep{Griffiths2017}. While essential large-scale
hydrodynamic relationships are relatively well-known for this region
\citep{deSteur2019}, information on the current distribution, abundance
and sensitivity to climate change is only partially known for a few
species (e.g., emperor penguin) \citep{Houstin2022}.

The network complexity of the Weddell Sea food web is high comprising
488 species and 16200 predator-prey interactions \citep{Jacob2011}. In
an attempt to better understand species roles related to food web
stability, Jacob et al.~(2011) performed secondary extinction
experiments and found that the removal of small to medium-sized, and not
large, organisms caused a cascade of secondary extinctions. This
findings highlighted the relative importance of predators, rather than
prey, for the architecture, functioning and stability of the Weddell Sea
food web, which coincides with findings from recent meta-analyses in
natural complex food webs \citep{Brose2019, Perkins2022}. Other
investigations considered this food web in a meta-analysis context
showing that high predator-prey body-mass ratios are found for predator
groups with specific trait combinations, including small vertebrates and
large swimming or flying predators \citep{Brose2019}. These trait
combinations generate weak interactions that stabilize communities
against perturbations maintaining ecosystem functioning.

In this study, we aim to go beyond a purely topological
(presence/absence) assessment of who eats whom in the Weddell Sea
ecosystem by providing a quantitative analysis of the trophic
interaction network. We aim to analyse the species' role for the
structure and stability of the food web. To achieve this, we: 1)
estimated the strength for each interaction in the Weddell Sea food web,
2) characterised species' role considering both weighted and unweighted
properties, and 3) analysed the species' role related to the stability
of the food web. This is the first time that interaction strengths were
estimated for all pairwise trophic interactions at the species level
(except for a few) for the Weddell sea food web.

\section{Methodology}

\subsection{Study area}

The high Antarctic Weddell Sea shelf is situated between 74 and 78ºS,
stretching approximately 450 km from East to West (Figure 1). Water
depth varies between 200 and 500 meters, and shallower areas are covered
by continental ice, which forms the coastline along the eastern and
southern part of the Weddell Sea. The shelf area contains a complex
three-dimensional benthic habitat with large benthic biomasses,
intermediate to high diversity in comparison to benthic boreal
communities and a spatially patchy distribution of organisms
\citep{Dayton1990, Teixido2002}.

\subsection{Weddell Sea food web dataset}

The Weddell Sea food web was retrieved from the GlobAL daTabasE of
traits and food Web Architecture (GATEWAy, version 1.0) of the German
Centre for Integrative Biodiversity Research (iDiv) Halle-Jena-Leipzig
\citep{Brose2018}. In addition to predator-prey interactions, the
database contains information on other biological data such as the mean
body mass and movement type for each species in the food web.
Furthermore, it incorporates information about the interaction itself,
such as the dimension of the predator search space (2 or 3 dimensions).
In its current form, the Weddell Sea food web comprises 490 species and
16041 predator-prey interactions and constitutes one of the most
resolved food webs constructed to date \citep{Jacob2011}.

\subsection{Dataset analyses}

\subsubsection{Interaction strength estimation and distribution}

To estimate the strength of each pairwise interaction in the food web,
we followed an approach proposed by \citet{Pawar2012}. The minimum data
requirements are body mass of the consumer (predator) and resource
(prey), and the interaction dimensionality (ID) classified as 2 or 3
dimensions. The ID is defined as the search space dimensions of the
predator, which is also equivalent to the movement space of the prey.
Thus, the ID is classified as 2D when both predator and prey move in 2D
(e.g., both are benthic) or if a predator moves in 3D and a prey in 2D
(e.g., pelagic predator on benthic prey). The ID is classified as 3D
when both predator and prey move in 3D (e.g., both pelagic) or if the
predator moves in 2D and the prey in 3D (e.g., benthic predator, pelagic
prey) \citep{Pawar2012}. GATEWAy v.1.0 provides information on the mean
body mass for consumers and resources, except for `detritus' and
`sediment', and the dimensionality for the majority of the interactions,
though the latter is missing in some cases (924 interactions). To
complete the missing data on species `dimensionality', we used
information about the movement type of predators and prey included in
GATEWAy.

The main equation we used for estimating the interaction strength IS
was:

\begin{equation}
IS = \alpha x_R \frac{m_R}{m_C}
\end{equation}

where \vec{\alpha} is the search rate, \vec{x_R} is the resource
density, and \vec{m_R} and \vec{m_C} are the body mass for the resource
and the consumer, respectively \citep{Pawar2012}.

We obtained estimates for resource density and the search rate from the
scaling relationships with the resource and the consumer mass,
respectively \citep{Pawar2012}. The coefficients of such relationships,
determined by ordinary least squares regression, vary with the
interaction dimensionality. On one hand, resource density scales with
resource mass as power-law with different exponents in 2D and in 3D.
Since mean mass for resources `phytodetritus' and `sediment' were not
available in GATEWAy, we considered the body mass of the smallest
phytoplankton species (`Fragilariopsis cylindrus') as a proxy. This is
justified by the fact that `phytodetritus' and `sediment' are mainly
composed of dead or senescent phytoplankton reaching the seabed
\citep{Wolanski2011}. On the other hand, search rate scales with
consumer mass as power-law with exponents in 2D and in 3D.

We fitted six candidate models (Exponential, Gamma, log-Normal, Normal,
Power-law and Uniform) the interaction strength distribution using
maximum likelihood \citep{McCallum2008}, and selected the best fitting
model by computing the Akaike Information Criterion AIC
\citep{Burnham2002}.

\subsubsection{Species properties}

To characterise the role of each species in the food web, we considered
unweighted and weighted food web properties (Figure 2). Unweighted
properties are related to properties commonly used in qualitative food
web studies and only describe the presence or absence of interactions
without any information on strength between a pairwise species link
\citep{Martinez1991, Dunne2002, Borrelli2014}. In contrast, weighted
properties capture the importance of a trophic interaction by
considering its strength.

To assess species roles as a function of the weighted food web, we
focused on mean interaction strength defined as the average strength of
all interactions for a given species. Further we calculated three
unweighted species properties: a) species degree, i.e., the sum of in-
and out-going interactions ; b) trophic level ; and c) trophic
similarity, i.e., the trophic overlap based on shared and unique
resources and consumers. These metrics were chosen to assess a species
role based on the unweighted network. The species degree has often been
equated with species importance to the structure and functioning within
a food web, i.e.~perturbations to high-degree species may therefore have
more significant effects on the food web robustness to perturbations
than low-degree species
\citetext{\citealp{Dunne2002a}; \citealp[references
in][]{Cirtwill2018a}}. The trophic level offers information about how
important a species is to its biotic community, i.e., top predators and
primary producers are expected to have particularly large effects on the
rest of their communities through top-down and bottom-up control,
respectively \citep[references in][]{Cirtwill2018a}. Trophic similarity
is an index of trophic overlap considering the set of prey and predators
for a pair of species; it measures one of the most important aspects of
species' niches, the trophic niche, and functional aspects of
biodiversity \citep{Martinez1991, Williams2000}.

Furthermore, we took species habitat affiliation into account, which
describes the physical position of a species within the ecosystem.
Species were categorised as: 1) benthic, if a species lives on the
seafloor; 2) pelagic, if a species lives close to the surface; 3)
benthopelagic, if it moves between and connects the aforementioned
environments; 4) demersal, if it lives and feeds on, or near, the bottom
of the sea; and 5) land-based, if the consumer is not strictly aquatic
but feeds predominantly on marine species. Species habitat affiliations
were retrieved from \citet{Jacob2011}.

To study the relationship between species mean interaction strength
(weighted property) and the unweighted species properties, we performed
linear regression analyses between the log mean interaction strength and
each of the aforementioned unweighted properties. Thus, we considered
the interaction strength as the dependent variable and the given
unweighted property as the independent variable, and obtained the
coefficients (slope and intercept) for the linear model. Models were
fitted using the least squares approach. We also explored the mean
interaction strength distribution with the species habitat.

Formulas used to obtain the above species properties are described in
Supplementary Material.

\subsubsection{Extinction simulations and stability}

To analyse the impact of species on food web stability, we performed
extinction simulations deleting one species at a time, that is for every
extinction, network size was reduced by one species only. After each
extinction, we calculated the stability of the network minus the removed
species (489 nodes) and compared it with that of the whole network (490
nodes in total). To calculate stability, we used the mean of the real
part of the maximum eigenvalue of the Jacobian matrix using randomized
Jacobians, keeping the predator-prey sign structure fixed
\citep{Allesina2008, Grilli2016}. This stability index indicates a more
stable food web when it is negative. We performed 1000 simulations for
each species removal and obtained a mean maximum eigenvalue for each
case. Finally, we statistically analysed this difference with an
Anderson-Darling test considering a p-value \textless{} 0.01
\citep{Scholz1987}. If the difference between the networks without and
with the removed species is positive, then the stability of the whole
food web is higher without the targeted species, in other words this
species makes the network less stable. If the difference is negative,
then the stability of the whole food web is lower without the targeted
species, i.e., this species ahs a stabilizing effect. A detailed
description on the stability calculations can be found in the
supplementary material.

To identify the species with the highest effect on food web stability,
we plotted the results of each species' extinction and its effect on
food web stability, considering weighted (interaction strength) and
unweighted properties, and species habitat.

All analyses were performed in R software, using the R packages igraph
\citep{Csardi2005}, cheddar \citep{Hudson2013}, and multiweb
\citep{Saravia2019}. The source code and data are available at
https://github.com/EcoComplex/WeddellSea.

\section{Results}

\subsection{Interaction strength distribution}

The statistical distribution that best fitted the empirical interaction
strength distribution of the Weddell Sea food web was a `gamma' due to
the high proportion of weak interactions and the existence of a few
strong interactions (Figure 3, Table S3).

\subsection{Species' role related to their mean interaction strength}

We found that the species' mean interaction strength (weighted property)
shows different relationships with the unweighted properties analysed
(Figure 4A-D). In this regard, there is a positive relationship between
interaction strength and trophic level, i.e., the higher the trophic
level of the species, the higher its mean interaction strength. We also
found a significant but less evident positive relationship with species
degree. Contrary, there was no significant relationship between mean
interaction strength and trophic similarity. Considering species habitat
affiliation, the ``Benthopelagic'' and ``Pelagic'' categories contained
the two species with the highest mean interaction strength, the killer
whale Orcinus orca and the colossal squid Mesonychoteuthis hamiltoni,
respectively. However, the majority of the species with relatively
higher interaction strength belonged to the ``Demersal'' and
``Land-based'' habitats groups. Species inhabiting the benthic realm
showed the lowest mean interaction strength (Figure 4D).

\subsection{Species impact on food web stability}

Our extinction analyses showed that the majority of species had no
significant impact on food web stability after being removed (Figure 5).
Most of the species (black points in figure 5) did not change the
stability of the network considerably after being removed, except for a
few species. Only 15 out of 490 species (3.06\%) gave rise to
significant changes in the food web's stability after their removal
(Table 2). Most of these species had a positive impact on food web
stability, i.e., network stability increased after their removal. Only
two species significantly decreased network stability after being
removed, the demersal fish Pagetopsis macropterus and the benthopelagic
amphipod Maxilliphimedia longipes.

After exploring the stability difference against the species properties
(Figure 5), we found that the species that generated a significant
impact on the stability of the food web were characterised by: 1) high
mean interaction strength; 2) mid to high trophic levels (TL
\textgreater{} 3.2); 3) relatively high number of interactions (Degree
\textgreater{} 25); and 4) mid to low trophic similarity (TS \textless{}
0.16). Habitat wise, species with a significant impact on the stability
were present in all habitats, except for the benthic realm. Table 2
shows the results for the species with highest impact on the food web
stability.

\section{Discussion}

\subsection{Many weak and a few strong interactions}

Our analyses show that the distribution of species interaction strength
(IS) at the network level is asymmetric, i.e., the Weddell Sea food web
contains many weak interactions and only a few strong ones. This finding
is consistent with many previous theoretical and empirical studies
\citep[e.g.][]{McCann1998, Neutel2002, Emmerson2004, Wootton2005, Kortsch2021}.
The asymmetric distribution of IS in food webs has been interpreted as
an explanation for the persistence of complex communities in nature
\citep{Bascompte2005, Allesina2015, Nilsson2016}. Here we show that this
pattern is also prevalent in the Weddell Sea, one of the most complex
food webs to date, comprising 490 species and 16041 predator-prey
interactions. This finding is in someway validating the method we used,
validation of allometric methods of IS estimation not including
interaction dimensionality has been performed for microcosmos of
relatively few species \citep{Jonsson2018}.

\subsection{Species's role related to their mean interaction strength}

We employed a range of descriptors using both unweighted and weighted
food web properties to characterise the dynamic and multifaceted nature
of the Weddell Sea food web. Our results show a positive relationship
between IS and trophic level, and between IS and species degree. In the
Weddell Sea, species with high degree also tend to have high mean ISs.
This positive relationship between IS and species degree reinforces the
central role of species with many interactions: species with a high
degree (hubs) have a large impact on overall food web structure and
functioning \citep{Dunne2002a, Kortsch2015}. On the other hand, the
positive IS-trophic level relationship contradicts studies that suggest
that mid-trophic level species (e.g.~krill, mesopelagic fish, squid) are
involved in the major pathways of energy flow in high-latitude marine
ecosystems
\citep{Pinkerton2014, Murphy2016, McCormack2020, Riccialdelli2020}. Such
contradiction could be explained by the lack of species biomass
information in the calculation of IS we applied here \citep{Pawar2012}.
Although this methodology allows information on species biomass or
density to be included, this type of data was not available for the
majority of species of the Weddell Sea food web.

Overall, the combination of information on the quantity and quality of
interactions and its relationship enables a robust assessment of the
species' role in the stability of the food web \citep{Cirtwill2018a}.

\subsection{Species impact on food web stability}

Only a few species play a key role with respect to the Weddell Sea food
web stability, according to the stability index employed in this study.
This is in concordance with other studies on complex empirical food webs
in marine ecosystems in the Arctic and other locations in Antarctica
\citep{Kortsch2015, Marina2018, Rodriguez2022}. These key species are
characterised by a particular set of food web properties: high to mean
IS; mid to high trophic level; a relatively high number of interactions;
and mid to low trophic similarity. In a previous study on sequential
extinction simulations for the Weddell Sea food web \citep{Jacob2011},
it was found that larger bodied-sized species could be lost without
causing a collapse of the network. A major caveat of this finding, also
recognised by the authors, was that population dynamics were ignored and
hence no top-down extinctions, or other indirect effects, could occur.
In our study we considered such top-down effects by including
information on the species IS, which is of paramount importance when
analysing the response of perturbations in ecological communities
\citep{McCann1998, Montoya2009, Novak2011}. Thus, our study suggests
that species with high mean IS and high trophic levels need to be
considered with particular attention when trying to predict the effects
of perturbations on the Weddell Sea ecosystem. This conclusion is
further reinforced by the finding that these species have mid to low
trophic similarity, which means that few other species of the food web
can occupy the same trophic role. In a review, it was emphasised that
polar pelagic communities are particularly sensitive to changes due to a
low functional redundancy at key trophic levels \citep{Murphy2016}. Here
we provide a broader analysis of the species impact on food web
robustness by including species from all habitats (benthic, pelagic and
land-based). This suggests that the sensitivity of marine polar
ecosystems to environmental perturbations is a concern also beyond the
pelagic realm.

\clearpage
\conclusions[Conclusions]

Our study goes beyond the current understanding of how species influence
ecosystem structure and stability in the Weddell Sea in particular, and
in most polar regions in general \citep{Murphy2016, McCormack2021}. In
the same analysis we integrated information on weighted (IS) and
unweighted species properties, enabling a more complete assessment of
the species' role realted food web structure and stability, and allowing
us to identify species and their characteristics which can have a
destabilising or stabilising effect on the food web.

We consider that the information provided in this study is important for
the development of effective policies and management strategies,
particularly given the ongoing initiative to implement a Marine
Protected Area (MPA) in the Weddell Sea region \citep{Teschke2021}.

\clearpage

\begin{figure}
\includegraphics[width=12cm]{Fig.1_StudyMap} \caption{Map of the Weddell Sea and Dronning Maud Land sector highlighting the high Antarctic shelf as a dashed-line contour. Modified from www.soos.aq.}\label{fig:unnamed-chunk-1}
\end{figure}

\clearpage

\begin{figure}
\includegraphics[width=12cm]{Fig.2_ToyFoodWeb} \caption{Scheme of a network showing the weighted and unweighted properties we used to characterize the species of the Weddell Sea food web. Directed arrows indicate the flow of energy; the width of the arrow represents the interaction strength of it.}\label{fig:unnamed-chunk-2}
\end{figure}

\clearpage

\begin{figure}
\includegraphics[width=12cm]{Fig3_IntDist} \caption{Frequency distribution of interaction strengths for the Weddell Sea food web. Total number of interactions = 16041. The distribution was best fitted to a ‘gamma’ model.}\label{fig:unnamed-chunk-3}
\end{figure}

\clearpage

\begin{figure}
\includegraphics[width=12cm]{Fig4_LinReg} \caption{Relationships between weighted (mean Interaction Strength) and unweighted properties including habitat. Linear regressions are shown between log(mean interaction strength) and trophic level (A), degree (B) and trophic similarity (C). Linear regressions for trophic level ($y = 1.12x - 15.29, R^2 = 0.43, p-value < 2e-16$), degree ($y = 0.006x - 12.77, R^2 = 0.03, p-value = 4.06e-5$) and trophic similarity ($y = -1.46x - 12.18, R^2 = -0.0004, p-value = 0.36$).}\label{fig:unnamed-chunk-4}
\end{figure}

\clearpage

\begin{figure}
\includegraphics[width=12cm]{Fig.5_QSSDif} \caption{Stability  difference (mean maximum eingenvalue) between the whole Weddell Sea food web (n = 490) and the food web minus one species (n = 489) for weighted (interaction strength) and unweighted species properties, and habitat. Point color indicates the impact on the stability; if significant the extinction of that species altered the stability of the food web.}\label{fig:unnamed-chunk-5}
\end{figure}

\clearpage

\begin{table}[t]
\caption{Properties of the species that when become extinct generated a significant impact on the stability of the Weddell Sea food web, ordered by significance (Anderson-Darling p-value). References: meanIS = mean interaction strength, TL = trophic level, Deg = degree, TS = trophic similarity, StabDif = stability difference, ADvalue = Anderson-Darling p-value.}
\begin{tabular}{l c c c c c c c}
\tophline

\textbf{Species} & \textbf{meanIS} & \textbf{TL} & \textbf{Deg} & \textbf{TS} & \textbf{Habitat} & \textbf{StabDif} & \textbf{ADvalue}\\
\middlehline
Orcinus orca & 1.83e-4 & 5.03 & 26 & 0.037 & Benthopelagic & 4.67e-5 & 2.28e-41 \\
\middlehline
Macrourus holotrachys & 8.30e-5 & 4.70 & 85 & 0.112 & Benthopelagic & 3.55e-5 & 2.73e-23 \\
\middlehline
Pagetopsis macropterus & 7.08e-5 & 4.64 & 76 & 0.113 & Demersal & -1.80e-5 & 2.38e-12 \\
\middlehline
Abyssorchomene nodimanus & 2.56e-5 & 4.21 & 137 & 0.130 & Benthopelagic & 2.30e-5 & 8.52e-10 \\
\middlehline
Dissostichus mawsoni & 7.82e-5 & 4.12 & 87 & 0.126 & Pelagic & 2.17e-5 & 1.57e-9 \\
\middlehline
Macrourus whitsoni & 7.14e-5 & 4.55 & 92 & 0.124 & Benthopelagic & 2.12e-5 & 3.30e-8 \\
\middlehline
Hydrurga leptonyx & 1.03e-4 & 4.72 & 67 & 0.094 & Land-based & 2.04e-5 & 9.66e-6 \\
\middlehline
Mesonychoteuthis hamiltoni & 1.80e-4 & 4.41 & 29 & 0.028 & Pelagic & 1.82e-5 & 4.59e-5 \\
\middlehline
Champsocephalus gunnari & 7.62e-5 & 3.72 & 46 & 0.086 & Pelagic & 1.83e-5 & 6.79e-5 \\
\middlehline
Notothenia marmorata & 8.27e-5 & 4.09 & 44 & 0.091 & Demersal & 1.60e-5 & 1.23e-4 \\
\middlehline
Arctocephalus gazella & 9.28e-5 & 4.67 & 61 & 0.093 & Land-based & 1.17e-5 & 2.09e-4 \\
\middlehline
Trematomus pennellii & 3.04e-5 & 4.04 & 192 & 0.158 & Demersal & 1.44e-5 & 1.00e-3 \\
\middlehline
Mirounga leonina & 1.20e-4 & 4.87 & 56 & 0.080 & Land-based & 1.41e-5 & 1.28e-3 \\
\middlehline
Notothenia coriiceps & 4.94e-5 & 4.27 & 130 & 0.126 & Demersal & 1.44e-5 & 1.66e-3 \\
\middlehline
Maxilliphimedia longipes & 2.21e-6 & 3.26 & 60 & 0.136 & Benthopelagic & -4.46e-6 & 9.74e-3 \\

\bottomhline
\end{tabular}
\end{table}

\appendixfigures
\clearpage

\begin{figure}
\includegraphics[width=12cm]{App1_FWplot} \caption{Graphic representation of the Weddell Sea food web. Species (nodes) are arranged vertically and colored by trophic level. The diameter of the node indicates the total number of interactions. Predator-prey interactions are represented by the arrows, from prey to predator.}\label{fig:unnamed-chunk-6}
\end{figure}






%%%%%%%%%%%%%%%%%%%%%%%%%%%%%%%%%%%%%%%%%%
%% optional

%%%%%%%%%%%%%%%%%%%%%%%%%%%%%%%%%%%%%%%%%%

%%%%%%%%%%%%%%%%%%%%%%%%%%%%%%%%%%%%%%%%%%
\authorcontribution{TIM and LAS: Conceptualization (lead); Data curation
(lead); Formal analysis (lead); Methodology (lead); Coding (lead);
Writing -- original draft (lead); Writing -- review and editing (lead).
SK: Conceptualization (lead); Formal analysis (supporting); Methodology
(supporting); Coding (supporting); Writing -- original draft
(supporting); Writing -- review and editing
(supporting).} %% optional section

%%%%%%%%%%%%%%%%%%%%%%%%%%%%%%%%%%%%%%%%%%
\competinginterests{The authors declare no competing
interests.} %% this section is mandatory even if you declare that no competing interests are present

%%%%%%%%%%%%%%%%%%%%%%%%%%%%%%%%%%%%%%%%%%

%%%%%%%%%%%%%%%%%%%%%%%%%%%%%%%%%%%%%%%%%%
\begin{acknowledgements}
Thanks to the rticles contributors!
\end{acknowledgements}

%% REFERENCES
%% DN: pre-configured to BibTeX for rticles

%% The reference list is compiled as follows:
%%
%% \begin{thebibliography}{}
%%
%% \bibitem[AUTHOR(YEAR)]{LABEL1}
%% REFERENCE 1
%%
%% \bibitem[AUTHOR(YEAR)]{LABEL2}
%% REFERENCE 2
%%
%% \end{thebibliography}

%% Since the Copernicus LaTeX package includes the BibTeX style file copernicus.bst,
%% authors experienced with BibTeX only have to include the following two lines:
%%
\bibliographystyle{copernicus}
\bibliography{../WeddellSea.bib}
%%
%% URLs and DOIs can be entered in your BibTeX file as:
%%
%% URL = {http://www.xyz.org/~jones/idx_g.htm}
%% DOI = {10.5194/xyz}


%% LITERATURE CITATIONS
%%
%% command                        & example result
%% \citet{jones90}|               & Jones et al. (1990)
%% \citep{jones90}|               & (Jones et al., 1990)
%% \citep{jones90,jones93}|       & (Jones et al., 1990, 1993)
%% \citep[p.~32]{jones90}|        & (Jones et al., 1990, p.~32)
%% \citep[e.g.,][]{jones90}|      & (e.g., Jones et al., 1990)
%% \citep[e.g.,][p.~32]{jones90}| & (e.g., Jones et al., 1990, p.~32)
%% \citeauthor{jones90}|          & Jones et al.
%% \citeyear{jones90}|            & 1990


\end{document}
